\chapter*{付録} % 章番号を出さない
\addcontentsline{toc}{chapter}{付録} % 目次に載せる

「付録」(appendix)は、論文の本文に載せるには情報として邪魔もしくは必須ではないものの、読者にとって有益となるような情報を載せます。付録を必要としない論文ももちろん存在しますので、そこは著者の判断です。

例えば、たくさんの観測データを様々なモデルでフィットした場合、フィット結果の絵がたくさん出てくるはずです。そのような図は本文中に大量に出されても大切な情報を見失ってしまいますので、大部分は付録に載せることが推奨されます。他には、何かしらの長い式変形や証明を載せる必要がある場合、付録に移動する場合があります。

% 付録は chapter の 1 つとして作りますが、章番号は表示しません。
% また付録の 1 つずつはアルファベットで番号付けをするのが一般的です。
\setcounter{section}{0} % section の番号をゼロにリセットする
\renewcommand{\thesection}{\Alph{section}} % 数字ではなくアルファベットで数える
\setcounter{equation}{0} % 式番号を A.1 のようにする
\renewcommand{\theequation}{\Alph{section}.\arabic{equation}}
\setcounter{figure}{0} % 図番号
\renewcommand{\thefigure}{\Alph{section}.\arabic{figure}}
\setcounter{table}{0} % 表番号
\renewcommand{\thetable}{\Alph{section}.\arabic{table}}

\section{すごい長い証明}
式~(\ref{eq})のように、式番号がアルファベットとアラビア数字の組み合わせになるように、\LaTeX{}ソース中で設定してありますので、中身を眺めてみてください。

\begin{equation}
  \label{eq}
  1 + 1 = 2
\end{equation}


\section{すごいたくさんのフィットの図}

\section{修士論文添削前に自己点検する項目}

\begin{itemize}
\item[\CID{00728}] 第\ref{chap:plagiarism}章を読み、剽窃について十分に理解したか。
\item[\CID{00728}] 修士論文に剽窃箇所もしくは剽窃と見なされうる箇所は存在しないか。
\item[\CID{00728}] \LaTeX\ で図番号などの参照先がないせいで「図??」「表??」「??節」のようになっている箇所はないか。
\item[\CID{00728}] 日本語読点「、」と欧文カンマ「,」が混在していないか。例えば「ガンマ線望遠鏡は、HESS, MAGIC, VERITASなどがある」。
\item[\CID{00728}] 日本語丸括弧「()」と欧文丸括弧「()」が混在していないか。
\item[\CID{00728}] 単位と数値の間にスペースは入っているか。「100MeV」など。
\item[\CID{00728}] 単位が斜体になっていないか。「$100~MeV$」など。
\item[\CID{00728}] 変数でない添字などが斜体になっていないか。$N_{trigger}$など。
\item[\CID{00728}] 自分で作成したものではない図や写真は、全て出典が明記され、転載であることを書いてあるか。
\end{itemize}
