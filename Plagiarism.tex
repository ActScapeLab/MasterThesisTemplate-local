\chapter{剽窃について}
\label{chap:plagiarism}
\section{剽窃とは何か}
\label{sec:plagiarism}
「剽窃(ひょうせつ)」とは辞書で次のように説明されています。
\begin{itemize}
\item 「他人の作品や論文を盗んで、自分のものとして発表すること。」『大辞泉』
\item 「他人の作品・学説などを自分のものとして発表すること。」『スーパー大辞林』
\item 「他人の著作から,部分的に文章,語句,筋,思想などを盗み,自作の中に自分のものとして用いること。他人の作品をそっくりそのまま自分のものと偽る盗用とは異なる。」『ブリタニカ国際大百科事典 小項目事典』
\end{itemize}

例えばここで『ブリタニカ国際大百科事典 小項目事典』を引用元として明記せずに、
\begin{quotation}
  \red{剽窃(ひょうせつ)とは、}他人の著作から\red{、}部分的に文章\red{、}語句\red{、}筋\red{、}思想などを盗み\red{、}自作の中に自分のものとして用いること\red{です}。他人の作品をそっくりそのまま自分のものと偽る盗用とは異な\red{ります}。
\end{quotation}
という説明をしたとします。これが剽窃です。この例では赤字で示したとおり、文体をですます調に変更したり、読点を「,」から「、」に変更したり、文頭に「剽窃(ひょうせつ)とは、」と書き加えたりしていますが、全体としては同一の文章であるため、通常は剽窃と見なされます。

学術論文ではない創作物の形態によっては、剽窃行為が「インスパイア」や「オマージュ」という言葉で括られることもあります。しかし修士論文での剽窃行為は不正行為です。試験でのカンニングやレポートの丸写しと同じであり、(まともな大学や研究室であれば)厳しく罰せられます。

\section{剽窃をするとどうなるか}

修士論文中に剽窃行為が発見された場合、その学期における単位をすべて没収され、卒業に必要な単位が与えられず修士課程を修了できなくなる可能性が高いです。各大学や研究科でどのような対応を実際に取るかはそれぞれだと思いますが、少なくとも私が審査員を担当した場合には落第させます。

修論審査に落第すれば、もし就職が決まっていても留年を余儀なくされます。留年を選択せず修了を諦めて中退するにしても、就職先は剽窃行為のせいで修了できなかった学生をそのまま採用はしてくれないでしょう。仮に同じ企業に就職が認められたとしても、修士卒扱いで入社できたはずのところが学部卒扱いとなり、初任給が月額数万円低い状態から開始となります。例えば同期と2万円の月給差を保ったまま40 年間働くとすると生涯収入で 1000 万円程度の損失になります。もし留年する道を選んでも、定年時点で1000万円程度の年収を見込めるのであれば、生涯収入としてその額だけ失うことになります。

もし博士課程に進学する場合、なぜ留年したかの説明を陰に陽に常に求められます。たとえ直接にその理由を問われることがなくとも、他の学生より1年多く修士課程に時間がかかったということは、優秀な学生ではないと周りから見なされ、研究をする上でも奨学金などを取得する上でも不利になるでしょう。また標準年限を超えての在籍の場合、大学院の授業料免除などの制度も利用できなくなる可能性があります。

\section{修士論文における剽窃について}
節\ref{sec:plagiarism}に引用した一般的な剽窃の定義ではなく、科学文書や、特に修士論文での剽窃についてもう少し踏み込んで説明し直してみましょう。

\subsection{いわゆるコピペ}
\label{subsec:plagiarism1}
少なくとも宇宙物理学分野における修士論文は独自性のあるものでなくてはいけません。独自性のある(オリジナル)とは次のようなことです。
\begin{itemize}
\item 誰かが過去にやった研究ではないこと
\item 自分自身の手でやった研究であること(共同研究であれば、十分に自分の貢献のあること)
\item 研究本体以外の章も含め、すべて自分の言葉で説明できること
\end{itemize}

したがって、誰かの論文や教科書の記述をそっくりそのまま持ってきて(いわゆる「コピペ」して)、それを自分の修士論文として提出することは許されません。高校や大学のレポートなどでも、他人のレポートを写すなと散々注意されるのと同じことです。

これはコピペする文章の長さに依りません。たとえ1行であってもコピペはコピペであり、剽窃と見なされます\footnote{ただし、ごくありふれた表現や、酷似するのが避けられない科学的事実は除く。}。

もちろん、ある文章を他の論文や書籍から引用(quote)する必要のある場合は、逆に改変してはいけません。そっくりそのまま書き写し、それを自分の文章とは別のものであると分かるように引用符や枠で囲むなりします。しかし宇宙物理学関連の修士論文でこのような引用をすることは、ほとんどないと思います。

\subsection{他人の文章の改変}
\label{subsec:plagiarism2}

コピペとともによく見られるのが、他人の文章を一部だけ改変して自分が書いたかのように装うことです。完全に同一のものを持ってくる方が簡単ですし、なぜこのような行動を取るのかよく分かりませんが、私の経験として最も多い剽窃行為がこの文章の一部改変です。

もしかすると「先輩の修論を写したりコピペするなよ。自分の言葉で書けよ」とだけ教員から指導を受けると、表面的に一部改変すれば剽窃にはならないと勘違いするのかもしれません。しかし元の文章が存在しなければ作成できないのですから、これは独自性のある文章とは見なされず、やはり剽窃行為となります。

たとえば次のような文章が「元ネタ」として存在していたとしましょう\footnote{これはきちんと添削を受けていない、今となっては恥ずかしい私の修論の一節ですが、あくまで例です。}。

\begin{quotation}
  1910年代にHessらによって宇宙線の存在が確認されて以来、様々なエネルギー領域、様々な検出器によって宇宙線の観測が行われてきた。同時に、ガリレオ以来発達してきた可視光による天体の観測も、電波望遠鏡や赤外望遠鏡の登場によって多波長での観測へと発展することとなった。

  宇宙線と言っても、その成分は電磁波、陽子、原子核、neutrinoなど様々であり、それらの持つエネルギーも広範にわたる。現在地球上で確認されている宇宙線のうち、最もエネルギーの高いものは$10^{20}$~eVを超える(最高エネルギー宇宙線)。これは人工的に到達できるエネルギーを実に8桁も上回るが、なぜそのような高エネルギーの宇宙線が存在するのかは謎である。加速機構、地球までの伝播過程、1次宇宙線成分は何であるのか、いずれも未解明のままであり、その興味は尽きない。
  \flushright{\citet{Okumura2005}より引用}
\end{quotation}

少しこれを改変してみましょう。赤字が削除箇所、青字が追加箇所です。実際に私が発見してきた剽窃行為には、このような改変が多くありました。
  
\begin{quotation}
\MyDIFdelbegin \MyDIFdel{1910年}\MyDIFdelend \MyDIFaddbegin \MyDIFadd{1912}\MyDIFaddend \MyDIFdelbegin \MyDIFdel{代}\MyDIFdelend \MyDIFaddbegin \MyDIFadd{年}\MyDIFaddend に Hess\MyDIFdelbegin \MyDIFdel{ら}\MyDIFdelend によって宇宙線\MyDIFdelbegin \MyDIFdel{の存在}\MyDIFdelend が\MyDIFdelbegin \MyDIFdel{確認}\MyDIFdelend \MyDIFaddbegin \MyDIFadd{初めて発見}\MyDIFaddend されて以来、\MyDIFdelbegin \MyDIFdel{様々な}\MyDIFdelend \MyDIFaddbegin \MyDIFadd{広い}\MyDIFaddend エネルギー\MyDIFdelbegin \MyDIFdel{領域}\MyDIFdelend \MyDIFaddbegin \MyDIFadd{範囲}\MyDIFaddend 、\MyDIFdelbegin \MyDIFdel{様々}\MyDIFdelend \MyDIFaddbegin \MyDIFadd{多種多様}\MyDIFaddend な検出器によって宇宙線\MyDIFdelbegin \MyDIFdel{の}\MyDIFdelend 観測が行われてきた。\MyDIFdelbegin \MyDIFdel{同時に}\MyDIFdelend \MyDIFaddbegin \MyDIFadd{また}\MyDIFaddend 、ガリレオ以来発達してきた可視光\MyDIFdelbegin \MyDIFdel{による天体の観測}\MyDIFdelend \MyDIFaddbegin \MyDIFadd{での天体観測}\MyDIFaddend も、電波望遠鏡や赤外望遠鏡\MyDIFaddbegin \MyDIFadd{という新しい観測手段}\MyDIFaddend の登場\MyDIFdelbegin \MyDIFdel{によって多波長での観測}\MyDIFdelend \MyDIFaddbegin \MyDIFadd{により、多波長観測}\MyDIFaddend へと発展\MyDIFdelbegin \MyDIFdel{することとなった}\MyDIFdelend \MyDIFaddbegin \MyDIFadd{した}\MyDIFaddend 。

宇宙線と\MyDIFdelbegin \MyDIFdel{言}\MyDIFdelend \MyDIFaddbegin \MyDIFadd{い}\MyDIFaddend っても、その成分は\MyDIFdelbegin \MyDIFdel{電磁波、}\MyDIFdelend 陽子、原子核、\MyDIFdelbegin \MyDIFdel{neutrino}\MyDIFdelend \MyDIFaddbegin \MyDIFadd{電子、ニュートリノ}\MyDIFaddend など様々であり、\MyDIFdelbegin \MyDIFdel{それらの持つ}\MyDIFdelend \MyDIFaddbegin \MyDIFadd{その}\MyDIFaddend エネルギー\MyDIFaddbegin \MyDIFadd{範囲}\MyDIFaddend も\MyDIFdelbegin \MyDIFdel{広範}\MyDIFdelend \MyDIFaddbegin \MyDIFadd{何桁}\MyDIFaddend に\MyDIFaddbegin \MyDIFadd{も}\MyDIFaddend わたる。現在 \MyDIFaddbegin \MyDIFadd{、}\MyDIFaddend 地\MyDIFdelbegin \MyDIFdel{球}\MyDIFdelend 上で確認されている宇宙線のうち、最もエネルギーの高いものは$10^{20}$~eVを超える(\MyDIFaddbegin \MyDIFadd{いわゆる}\MyDIFaddend 最高エネルギー宇宙線)。これは\MyDIFdelbegin \MyDIFdel{人工的に}\MyDIFdelend \MyDIFaddbegin \MyDIFadd{加速器で人類が}\MyDIFaddend 到達できるエネルギーを\MyDIFdelbegin \MyDIFdel{実に}\MyDIFdelend 8桁も上回るが、なぜそのような高\MyDIFaddbegin \MyDIFadd{い}\MyDIFaddend エネルギーの宇宙線が存在するのかは\MyDIFdelbegin \MyDIFdel{謎である}\MyDIFdelend \MyDIFaddbegin \MyDIFadd{解明されていない}\MyDIFaddend 。\MyDIFaddbegin \MyDIFadd{宇宙線の}\MyDIFaddend 加速機構、地球までの伝播過程、\MyDIFaddbegin \MyDIFadd{また}\MyDIFaddend 1次宇宙線成分は何であるのか\MyDIFaddbegin \MyDIFadd{は}\MyDIFaddend 、いずれも未解\MyDIFdelbegin \MyDIFdel{明}\MyDIFdelend \MyDIFaddbegin \MyDIFadd{決}\MyDIFaddend の\MyDIFdelbegin \MyDIFdel{まま}\MyDIFdelend \MyDIFaddbegin \MyDIFadd{問題}\MyDIFaddend であり、\MyDIFdelbegin \MyDIFdel{その興味は尽きない}\MyDIFdelend \MyDIFaddbegin \MyDIFadd{将来の宇宙線観測計画による解決が期待される}\MyDIFaddend 。
  \flushright{\citet{Okumura2005}を意図的に改変}
\end{quotation}

\subsection{元の文章を下敷きに自分で考えたつもりになったもの}
\label{subsec:plagiarism3}

さらに改変の量を増やし、ところどころに自分の独自の文を入れたり、文の前後を入れ替える剽窃もあります。自分で考えて文を挿入するのだから剽窃ではないと考える人もいるかもしれませんが、やはり元の文章が存在しなければ書くことのできない文章ですので、これも立派な剽窃です。たとえば次のようなものです。

\begin{quotation}
  Hessの気球実験によって1912年に宇宙線が大気中で発見されてから、様々な粒子、多様な検出手法、またMeV領域から$10^{20}$~eVにまでおよぶエネルギー範囲で宇宙線の観測が行われてきた。一方、電磁波による天体の観測も、ガリレオによる可視光観測に始まり、電波望遠鏡や赤外線望遠鏡などの登場によって他波長観測へと発展した。さらに近年の重力波やニュートリノ観測を加え、現在の宇宙観測は、多粒子、他波長観測の時代、すなわちマルチメッセンジャー天文学へと進展した。

  このうち宇宙線は、陽子、原子核、電子、ニュートリノなどを含む、宇宙空間を飛び交う高エネルギーの粒子である。先に述べたように、その最高エネルギーは$10^{20}$~eVにまでわたる(いわゆる最高エネルギー宇宙線)。これは人類がLHC加速器で到達できる数~TeVというエネルギーを8桁も上回るものであるが、なぜそのような高いエネルギーの宇宙線が宇宙で加速されているのか、宇宙線の発見から100年以上が経っても未解決の問題である。その加速機構、加速天体、地球までの伝播、また粒子の種類がなんであるかという謎を解き明かすには、今後の宇宙線観測手法に大きな飛躍が必要である。
  \flushright{\citet{Okumura2005}を意図的に改変}
\end{quotation}

ここまで改変すると、全く違う文章のように感じる人もいるかもしれませんが、実際に行われる剽窃行為では、このような元ネタに改変を加えた文章が何段落も続くことが多いです。そのため、文章の一部が似通っているだけでなく、その章の論理展開自体がほとんど同じになってしまうのです。

研究背景は過去に行われた研究の積み重ねなので、論理展開が同じになることは仕方がないという主張をする学生もいます。しかし修士論文はその研究目的が各々違うわけですから、論文のイントロなどで全く同じ論理展開になることは本来ありえません。その論文独自の研究内容を説明するためにイントロは書かれるべきであり、他の文章と同じであるというのは、イントロを書くという目的を勘違いしています。

\subsection{出典のない図表の使用}

他人の文章を剽窃する行為とは別に、図表を適切に引用(cite)せずに流用するという剽窃もあります。これは悪意があって行われているわけではなく、引用の作法を知らないだけの場合が多いため罪としては軽いかもしれません。しかし、その修士論文の読者に対して「この図は自分が作りました」と嘘をつくのと同じ行為ですので、やはり問題行為であることは理解できると思います。

このような図表の剽窃は、特に共同研究で多く見られます。ある実験プロジェクトに参加している場合、実験装置の説明の図や写真をプロジェクト内で使いまわすことがあるでしょう。たとえば図\ref{fig_CTA}のようなものが該当します。もしこれを出典もしくは作者を明記せずに使用した場合、剽窃行為に当たります\footnote{おそらく「出典を明記して再提出しろ」と言われるだけで、落第はしないと思いますが。}。

図表の提供者の名前を入れる、その図が最初に使われた論文や出版物が存在する場合はそれを出典として明記する(cite する)、もしくは提供した実験グループなどの名前を入れるなどしてください。

\subsection{アイデアの盗用}
他人の考えた研究アイデアを自分が考えたかのように記述するのも剽窃です。例えば投稿論文になっていないものの、先輩の修士論文で先行研究が行われていたとしましょう。これを先行研究として取り上げることなく、「〜〜という手法を本論文では考案し」などと書くのは剽窃行為です。きちんと「〜〜という手法が先行研究で提案され、本論文ではこれを発展させ」のように書きましょう。

\subsection{自己剽窃}

自己剽窃とは、自分の書いた論文などから図や文章を剽窃して再利用することです。なぜこれが問題とされるのか、直感的にはすぐに分からないかもしれません。

自己剽窃が最も問題とされるは、論文の二重投稿です。どこかで論文を出版する場合、レビュー論文でない限り、それぞれが独自の新規性を持つ論文でなくてはいけません。したがって、業績稼ぎのために同じ内容の論文を複数の場所で発表するのは研究不正として扱われます。

次に自己剽窃が問題となるのは、著作権の問題です。投稿論文を科学誌に掲載する多くの場合、その著作権を出版社に譲渡することになります。最近のオープンアクセス(open access)誌の場合には著作権が論文著者に残される場合もありますが、投稿論文の著作権を必ずしも自分が持っているわけではないのだということを覚えておいてください。

著作権が出版社にあるということは、その著作物を引用の範囲を超えて勝手に再利用してはいけないということになります。著作権、英語で書くと copyright ですが、すなわち複製する権利を出版社に譲渡してしまっているからです。

ただし、多くの出版社では学位論文や国際会議のプロシーディングスなどで、著者が図表などを出版社に断らずに使いまわすことを許可しています。ただし、出典を明記することは求められていることが多いはずです。もし投稿論文に使用した図表もしくは文章を修士論文で使いまわす場合、出版社との著作権の契約について理解しておきましょう。たとえば Elsevier 社の場合、\url{http://jp.elsevier.com/authors/author-rights-and-responsibilities} に著者の権利が書かれています。他の出版社も同様の情報を公開しています。

\section{剽窃行為と実際に認定された例}

節~\ref{subsec:plagiarism1}や節~\ref{subsec:plagiarism2}では私が改変の例を示しましたが、これは本文書のために意図的に用意したものにすぎませんので、あまり現実味がありません。そこで、剽窃の含まれる書籍を実際に研究者が出版した例を見てみましょう。これは東洋英和女学院大学の教授が行った剽窃であり、大学はこの研究不正に対して懲戒解雇処分を決定しています\footnote{\url{http://www.toyoeiwa.ac.jp/daigaku/news/topics/news_2019051001.html}}。

以下に、この研究不正の調査対象となった『ヴァイマールの聖なる政治的精神-ドイツナショナリズムとプロテスタンティズム』(深井智朗、岩波書店、2012年)で見つかった剽窃行為の一部を引用します\footnote{元の比較は\url{http://www.toyoeiwa.ac.jp/daigaku/news/news_201905100104.pdf}で公開されています。}\footnote{実際に見つかった研究不正はこれ以外にもありますが、本書では割愛します。}。下線は\ured{同一の表現を用いている部分}、\ublue{同様の内容で表現が異なる部分}と別れています。

\begin{quotation}
  レーフラーも、ニーチェのキリスト教批判がその矛先を向けているのは、カントの影響を受け、 \ured{神} を \ured{実践理性の要請として理解} し、\ured{キリストの神性は宗教的な価値判断} であると考えたリッチェル学派、とりわけヴィルヘルム・ヘルマン的なキリスト教の再構築にあると見ている。そこでは既に述べた通り、 \ured{人間の意志が行う価値評価がキリスト教信仰の} \ublue{生みの母} であると理解されているので、 \ured{リッチェル学派} はニーチェのキリスト教批判に対して \ublue{完全に} \ured{無防備であ} り、逆に \ured{この神学に対して} は \ured{ニーチェのあらゆる価値の転倒というプログラムは完全な破壊力を持っていた} というのである。つまり、 \ured{この神学は} 、 \ured{ニーチェの前提、すなわち宗教的言表は価値評価をする意志による} \ublue{決断} \ured{であるという前提を基盤として} 成り \ured{立っている} のであり、ただリッチェル学派は \ured{ニーチェとは逆} の \ublue{結論を} 出 \ured{しただけ} なのである。それ故にレーフラーはヴィルヘルム期に神学者たちをニーチェと共にトータルに否定することができたのである。
  \flushright{深井智朗『ヴァイマールの聖なる政治的精神-ドイツナショナリズムとプロテスタンティズム』\\(岩波書店、2012 年 p.197)\\※剽窃先です}
\end{quotation}

\begin{quotation}
この学派においては---リッチェルよりもむしろヴィルヘルム・ヘルマンにおいて一層そうだったのだが---、 \ured{神} は \ured{実践理性の要請として理解} され、\ured{キリストの神性は宗教的な価値判断} として理解された。それによって \ured{人間の意志が行なう価値評価がキリスト教信仰の} \ublue{母胎} であることが解明されたのである。すでにトレルチは、宗教を幻想とみなすフォイエルバッハの診断に対して、この \ured{リッチェル学派} の神学が \ublue{全くの} \ured{無防備であ}ることを強調したわけであるが、さらに \ured{この神学に対してニーチェのあらゆる価値の転倒というプログラムは} 、\ured{完全な破壊力を持っていた} わけである。\ured{この神学は} まさにそれ自体 \ured{ニーチェの前提、すなわち宗教的言表は価値評価をする意志による} \ublue{判断} \ured{であるという前提を基盤として} その上に \ured{立っている}。この神学はただ \ured{ニーチェとは逆} に \ublue{価値評価を} \ured{しただけ} である。
  \flushright{W.パネンベルク著 近藤勝彦・芳賀力訳『組織神学の根本問題』\\(日本基督教団出版局、1984 年 p.277)\\※元の文献です}
\end{quotation}

\section{故意か過失か}
剽窃行為が行われたと認定された場合、本人が故意で(すなわち悪いと気が付いていて意図的に)剽窃を行ったか、それとも悪意はなく、誰かの文章を「参考」にしていたら結果として偶然に剽窃のようになってしまったか、その見極めは誰にもできません。したがって、この区別は剽窃行為の有無の判断には関係がありません。言い換えれば、結果として過失で剽窃に見えるような文章を作り出してしまった場合、それは剽窃と見なされます。剽窃だと見なされないような文章を自分の言葉で作成するのは、修士論文執筆者の責任です。

\section{剽窃かどうかの判断基準}

剽窃かどうかの判断基準は、どこかで明確に線引きをできるものではありません。例えば節\ref{subsec:plagiarism3}で例示したものが剽窃に当たるかどうか、その判断は教員によっても分かれるところでしょう。

学術論文出版社の Elsevier 社が、剽窃を行ったかどうかをどのようにして判断すれば良いのか、その基準を書いています\footnote{https://www.elsevier.com/editors/perk/plagiarism-complaints}。これも全ての修士論文や全ての分野に当てはまるか判断の難しい場合もあるかもしれませんが、考え方の参考になると思います。

Elsevierの説明では「Literal copying」(逐語的な複製、本書の節\ref{subsec:plagiarism1}に該当)、「Substantial copying」(部分的な複製、節\ref{subsec:plagiarism2})、「Paraphrasing」(言い換え、節\ref{subsec:plagiarism3})に分かれています。このうち判断基準が曖昧となりやすい、後者2つについての判断基準をElsevierの説明から抜粋引用します(強調は筆者による)。

\begin{quote}
\textbf{Substantial copying}\\
(略)\\
In addition to judging the quantity and quality of the copied content, you should consider the following question: \textbf{Has the author benefited from the skill and judgment of the original author?} The degree to which the answer to this question is “yes” will indicate whether substantial copying has taken place.

\textbf{Paraphrasing}\\
(略)\\
\textbf{Look at the quantity and quality of what has been taken and also at whether the second author has benefited from the skill and judgment of the first author. If it seems clear, on a balance of probabilities}, that the second author has taken without permission or acknowledgment all or a substantial part of the original work and used it to create a second work, albeit expressed in different words, then such use amounts to plagiarism.

\begin{flushright}
  Elsevier 社「Plagiarism complaints」より
\end{flushright}
\end{quote}

少し日本語に直すと、前者の「Substantial copying」の判断基準として、修士論文の執筆者が元文献著者のskill(作文技術)やjudgement(意見、見解)から恩恵を受けているかと、自身に問うてみよということです。誰が書いてもほぼ同じにしかならないような一般的な事実や短文\footnote{例えば、「地球とは太陽系の第三惑星である」など。}を除いて、元ネタを参考にしないと書けないような文章かどうかということです。ここで第三者に対して「これは参考とした文献の言葉を借りずに自分で書いたものだ」と強く主張できないようであれば、剽窃と見なされうるということを理解してください。

「Paraphrasing」については、やはり同様に元ネタのskillとjudgementから恩恵を受けているかどうかで判断しろとあり、これが程度問題で明白であれば、言葉を言い換えていたとしても剽窃であると考えるとしています。

\section{なぜ剽窃は許されないのか}

なぜ剽窃行為は許されず、それが修士論文で不正行為とされるのか、その理由を改めてまとめます。

\begin{enumerate}
\item 学位審査は、学生が研究背景などを理解しているか、またそれを自分の言葉で伝える能力を身につけているかを審査する場です。したがって、剽窃を含む文書ではこの審査を適切に行えなくなってしまいます。修士の学位を与える審査の一環として修士論文を執筆しているわけですから、修士論文作成能力がないのにそれを他人の文章を使って誤魔化すのは、当然不正行為になります。

\item 同じ文章を使いまわすとき、一般的には引用 (cite ではなくて quote) をし、自分の書いた文章と他人の文章を区別するのが標準的です。超新星の過去の記録など一部の例を除き、宇宙物理学分野でquoteのほうの引用をすることはほとんどありません。もし必要となる場合は、他人の書いた文章であることが明確に読者に分かるようにしましょう。自分で作った文章かのように見せるのは決して許される行為ではありません。

\item 他人の書いた文章を自分が書いたかのように見せるのは、人の手柄を横取りすることになります。

\item 少なくとも日本の国内においては、他人の著作物を勝手に使用したり改変したりすることは、著作権の侵害に当たる行為です。

\item 元の文章を無理に改変することにより、推敲された元の文章よりも質の低い文章になることが多く、また間違った記載となる場合が多々あります。例えば「突発天体を観測する」を無理やり「突発天体を監視する」に変更することにより、意味が大きく変わることもあります。

\item 同じものを繰り返すというのは、先人の研究をさらに発展させていくという、科学の営み自体を否定する行為です。

\item 過去数年で該当分野に大きな進展があった場合にも、それを無視した様な文章が生産されてしまいます。例えば 2018 年の修論なのに重力波が未だ検出されていない前提の文章になっていたりということが考えられます。

\item 修論の添削をする教員は、執筆した学生の研究能力や文章作成能力を高めるために添削をしています。良い出来の修論を書かせることが目的ではないのです。そのため、本人が書いてすらいない文章を添削させ、大学教員の貴重な時間を奪うことは、学生と教員の間の信頼関係を大きく毀損する大変失礼な行為です。またそのような添削をしても本人が書いていないのですから、その学生の能力向上には全く役に立たず、学生も自分で考えることなく言われるがままに改訂を繰り返すことになるでしょう。

\end{enumerate}
